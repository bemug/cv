%% Copyright 2006-2013 Xavier Danaux (xdanaux@gmail.com).

\documentclass[10pt,a4paper,sans]{moderncv}        % possible options include font size ('10pt', '11pt' and '12pt'), paper size ('a4paper', 'letterpaper', 'a5paper', 'legalpaper', 'executivepaper' and 'landscape') and font family ('sans' and 'roman')

% moderncv themes
\moderncvstyle{casual}                             % style options are 'casual' (default), 'classic', 'oldstyle' and 'banking'
\moderncvcolor{blue}                               % color options 'blue' (default), 'orange', 'green', 'red', 'purple', 'grey' and 'black'
%\renewcommand{\familydefault}{\sfdefault}         % to set the default font; use '\sfdefault' for the default sans serif font, '\rmdefault' for the default roman one, or any tex font name
%\nopagenumbers{}                                  % uncomment to suppress automatic page numbering for CVs longer than one page

% character encoding
\usepackage[utf8]{inputenc}                       % if you are not using xelatex ou lualatex, replace by the encoding you are using
%\usepackage{CJKutf8}                              % if you need to use CJK to typeset your resume in Chinese, Japanese or Korean

% adjust the page margins
\usepackage[scale=0.85]{geometry}
%\setlength{\hintscolumnwidth}{3cm}                % if you want to change the width of the column with the dates
%\setlength{\makecvtitlenamewidth}{10cm}           % for the 'classic' style, if you want to force the width allocated to your name and avoid line breaks. be careful though, the length is normally calculated to avoid any overlap with your personal info; use this at your own typographical risks...

% personal data
\name{Benjamin}{Mugnier}
\title{}                               % optional, remove / comment the line if not wanted
\address{25 Impasse Chez Tounin}{74150}{Moye}% optional, remove / comment the line if not wanted; the "postcode city" and "country" arguments can be omitted or provided empty
\phone[mobile]{+33~(0)6~16~50~70~75}                   % optional, remove / comment the line if not wanted; the optional "type" of the phone can be "mobile" (default), "fixed" or "fax"
%\phone[fixed]{+2~(345)~678~901}
%\phone[fax]{+3~(456)~789~012}
\email{mugnier.benjamin@gmail.com}                               % optional, remove / comment the line if not wanted
\homepage{bemug.github.io}                         % optional, remove / comment the line if not wanted
\social[linkedin]{Benjamin Mugnier}                        % optional, remove / comment the line if not wanted
\social[twitter]{@totallyNotBemug}                             % optional, remove / comment the line if not wanted
\social[github]{github.com/bemug}                              % optional, remove / comment the line if not wanted
%\extrainfo{additional information}                 % optional, remove / comment the line if not wanted
%\photo[64pt][0.4pt]{picture}                       % optional, remove / comment the line if not wanted; '64pt' is the height the picture must be resized to, 0.4pt is the thickness of the frame around it (put it to 0pt for no frame) and 'picture' is the name of the picture file
%\quote{Some quote}                                 % optional, remove / comment the line if not wanted

% to show numerical labels in the bibliography (default is to show no labels); only useful if you make citations in your resume
%\makeatletter
%\renewcommand*{\bibliographyitemlabel}{\@biblabel{\arabic{enumiv}}}
%\makeatother
%\renewcommand*{\bibliographyitemlabel}{[\arabic{enumiv}]}% CONSIDER REPLACING THE ABOVE BY THIS
\renewcommand*{\namefont}{\fontsize{32}{34}\mdseries\upshape}

% bibliography with mutiple entries
%\usepackage{multibib}
%\newcites{book,misc}{{Books},{Others}}
%----------------------------------------------------------------------------------
%            content
%----------------------------------------------------------------------------------
\begin{document}
%\begin{CJK*}{UTF8}{gbsn}                          % to typeset your resume in Chinese using CJK
%-----       resume       ---------------------------------------------------------
\vspace*{-1.3cm}
\makecvtitle

%Too much space over there
\vspace*{-1.5cm}

\section{Études}
\cventry{2012--2015}{Elève ingénieur}{ENSIMAG}{Grenoble}{}{École Nationale
	Supérieure d'Informatique et de Mathématiques Appliquées de Grenoble}  % arguments 3 to 6 can be left empty
\cventry{2010--2012}{DUT Informatique}{IUT d'Annecy}{Annecy-Le-Vieux}{}{}

\section{Expérience}
\subsection{Informatique}
\cventry{2015 (6 mois)}{Ingénieur
 	stagiaire}{Kalray}{Montbonnot-Saint-Martin}{Développement de supercalculateurs sur puces}
 	{Implémentation du standard Unix Device Tree (description du hardware) dans les outils Kalray
 	\begin{itemize}
 		\item Développement d'une bibliothèque permettant de manipuler le device
 			tree de façon transparente pour l'utilisateur
 		\item Adaptation du bootloader pour précharger le device tree dans la
 			mémoire interne
 		\item Modification des drivers PCIe/JTAG pour charger à la volée le
 			device tree
 		\item Modification du simulateur pour émuler la lecture de device tree
 	\end{itemize}
 	}
\cventry{2014 (3 mois)}{Ingénieur stagiaire}{ST
	Microelectronics}{Crôlles}{Fabricant de semi-conducteurs}
	{GreenNet : Capteurs énergétiquement autonomes dialoguant avec Internet
		(IoT)
	\begin{itemize}
		\item Développement d'un protocol permettant l'initialisation réseau des
			capteurs
		\item Permettre le relevé d'information sur l'état des noeuds routeurs
		\item Amélioration du SDK développeur
	\end{itemize}
	}
\cventry{2012 (6 mois)}{Programmeur}{Laboratoire
	LISTIC}{Annecy-Le-Vieux}{Laboratoire de recherches en informatique}
	{Listic Demonstrator : interface web permettant aux chercheurs de stocker des
	données satellitaires
	\begin{itemize}
		\item Permettre à l'utilisateur de charger des images satellitaires sur
			le serveur
		\item Sauvegarder sa méta-information dans une base de données
		\item Effectuer des traitements divers sur les images selon une chaîne
			d'opération markovienne
		\item Sécuriser le serveur
	\end{itemize}
	}
\cventry{2012 (6 mois)}{Programmeur chef de projet}{Laboratoire
	LISTIC}{Annecy-Le-Vieux}{Laboratoire de recherches en informatique}
	{JKDTT : Gestionnaire d'ontologies pour le modèle Knowledge Dependant Type
		Theory
	\begin{itemize}
		\item Charger des ontologies sous leur format propre (OWL)
		\item Afficher l'ontologie sous forme de graphe
		\item Permettre à l'utilisateur de modifier éléments et liens de
			l'ontologie
		\item Sauvegarder l'ontologie dans un fichier, le cloud, ou en imprimer
			un résumé
	\end{itemize}
	}
\subsection{Divers}
\cventry{Etés 2012 et 2013}{Jardinier}{Albanais Home Services}{Moye}{}{Petits
	travaux et aide à la personne}

\section{Compétences}
\cvitem{Scripting}{Ada, Bash, Vim script, Source Engine, Python 3, Ruby}
\cvitem{Objet}{Java, C++, C\#, VB.net, AS 3.0, UML, notions d'IA}
\cvitem{Bas niveau}{C, ASM (x86, x86\_64, MIPS), VHDL, PCIE, JTAG, Drivers
	Linux, Linker, notions d'émulation}
\cvitem{Moteurs}{Source Engine, Unity3D, Warcraft 3 editor}
\cvitem{APIs/Libs}{SFML, SDL, OpenGL}
\cvitem{SGBD}{SQL (Oracle, PostGres, MySQL), Modèles relationnel (UML, Merise)}
\cvitem{Utilitaires}{Git, SVN, Make, CMake, LaTex, Microsoft Office}
\cvitem{Design}{3DS Max, Photoshop, Flash, Premiere, After Effects}

\section{Langues}
\cvitem{Anglais}{Lu parlé écrit (TOEIC 950)}
\cvitem{Italien}{Intermédiaire}

\section{Centre d'intérêts}
\cvitem{Musique}{Pratique de la guitare}
\cvitem{Programmation}{Développement de jeux vidéo amateurs/modding}
\cvitem{Jeux vidéo}{Jeux solo, multijoueurs, compétitifs et retro gaming}


\end{document}
