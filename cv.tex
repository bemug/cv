%% Copyright 2006-2013 Xavier Danaux (xdanaux@gmail.com).

\documentclass[11pt,a4paper,sans]{moderncv}        % possible options include font size ('10pt', '11pt' and '12pt'), paper size ('a4paper', 'letterpaper', 'a5paper', 'legalpaper', 'executivepaper' and 'landscape') and font family ('sans' and 'roman')

% moderncv themes
\moderncvstyle{casual}                             % style options are 'casual' (default), 'classic', 'oldstyle' and 'banking'
\moderncvcolor{blue}                               % color options 'blue' (default), 'orange', 'green', 'red', 'purple', 'grey' and 'black'
%\renewcommand{\familydefault}{\sfdefault}         % to set the default font; use '\sfdefault' for the default sans serif font, '\rmdefault' for the default roman one, or any tex font name
%\nopagenumbers{}                                  % uncomment to suppress automatic page numbering for CVs longer than one page

% character encoding
\usepackage[utf8]{inputenc}                       % if you are not using xelatex ou lualatex, replace by the encoding you are using
%\usepackage{CJKutf8}                              % if you need to use CJK to typeset your resume in Chinese, Japanese or Korean

% adjust the page margins
\usepackage[scale=0.85]{geometry}
%\setlength{\hintscolumnwidth}{3cm}                % if you want to change the width of the column with the dates
%\setlength{\makecvtitlenamewidth}{10cm}           % for the 'classic' style, if you want to force the width allocated to your name and avoid line breaks. be careful though, the length is normally calculated to avoid any overlap with your personal info; use this at your own typographical risks...

% personal data
\name{Benjamin}{Mugnier}
\title{Embedded Software Engineer}                               % optional, remove / comment the line if not wanted
\address{34 Rue de la Biscuiterie}{38400}{Saint Martin d'Hères}% optional, remove / comment the line if not wanted; the "postcode city" and "country" arguments can be omitted or provided empty
\phone[mobile]{+33~(0)6~16~50~70~75}                   % optional, remove / comment the line if not wanted; the optional "type" of the phone can be "mobile" (default), "fixed" or "fax"
%\phone[fixed]{+2~(345)~678~901}
%\phone[fax]{+3~(456)~789~012}
\email{mugnier.benjamin@gmail.com}                               % optional, remove / comment the line if not wanted
%\homepage{bemug.github.io}                         % optional, remove / comment the line if not wanted
\social[linkedin]{benjamin-mugnier}                        % optional, remove / comment the line if not wanted
\social[github]{bemug}                              % optional, remove / comment the line if not wanted
%\extrainfo{additional information}                 % optional, remove / comment the line if not wanted
\photo[64pt][0pt]{img/photo_s.jpg}                       % optional, remove / comment the line if not wanted; '65pt' is the height the picture must be resized to, 0.4pt is the thickness of the frame around it (put it to 0pt for no frame) and 'picture' is the name of the picture file
%\quote{Some quote}                                 % optional, remove / comment the line if not wanted

% to show numerical labels in the bibliography (default is to show no labels); only useful if you make citations in your resume
%\makeatletter
%\renewcommand*{\bibliographyitemlabel}{\@biblabel{\arabic{enumiv}}}
%\makeatother
%\renewcommand*{\bibliographyitemlabel}{[\arabic{enumiv}]}% CONSIDER REPLACING THE ABOVE BY THIS

% bibliography with mutiple entries
%\usepackage{multibib}
%\newcites{book,misc}{{Books},{Others}}
%----------------------------------------------------------------------------------
%            content
%----------------------------------------------------------------------------------
\begin{document}
%\begin{CJK*}{UTF8}{gbsn}                          % to typeset your resume in Chinese using CJK
%-----       resume       ---------------------------------------------------------
\vspace*{-1cm}
\makecvtitle

%Too much space over there
%\vspace*{-2cm}
\vspace*{-1.5cm}
\setlength{\hintscolumnwidth}{0.15\textwidth} %Increse date column width

\section{Experiences}
\subsection{Computer science}
	\cventry{2015 (5 years)}{Embedded systems software engineer}{Kalray}{Grenoble}{Semiconductor company}
	{
	\begin{itemize}
		\item Linux Ethernet driver for kvx boards:
		\begin{itemize}
			\item Phy/MAC/PCS/Ethernet blocks programming for
				Ethernet communication.
			\item Auto-negociation and Link Training for plug and
				play interoperability.
			\item Receive Side Scaling implementation for packet
				processing offload.
		\end{itemize}
		\item Linux DMA Engine framework implementation for kvx core.
		\item Linux drivers and bare libraries for Network on Chip
			communication.
		\item Device tree and Linux FDT library bare mode port allowing
			board agnostic executables.
		\item Hypervisor/Operating System programming.
	\end{itemize}
	}
\cventry{2014 (3 months)}{Engineering intern}{ST Microelectronics}{Grenoble}{Semiconductor company}
	{Internet energetic autonomous sensors (IoT)
	\begin{itemize}
		\item Sensor network initialization protocol.
		\item Nodes states information harvesting.
		\item SDK optimization.
	\end{itemize}
	}

%\cventry{2012 (6 mois)}{Programmeur}{Laboratoire
%	LISTIC}{Annecy-Le-Vieux}{Laboratoire de recherches en informatique}
%	{Listic Demonstrator : Interface web permettant aux chercheurs de stocker des
%	données satellitaires
%	\begin{itemize}
%		\item Permettre à l'utilisateur de charger des images satellitaires sur
%			le serveur
%		\item Sauvegarder sa méta-information dans une base de données
%		\item Effectuer des traitements divers sur les images selon une chaîne
%			d'opération markovienne
%		\item Sécuriser le serveur
%	\end{itemize}
%	}
%\cventry{2012 (6 mois)}{Programmeur chef de projet}{Laboratoire
%	LISTIC}{Annecy-Le-Vieux}{Laboratoire de recherches en informatique}
%	{JKDTT : Gestionnaire d'ontologies pour le modèle Knowledge Dependant Type
%		Theory
%	\begin{itemize}
%		\item Charger des ontologies sous leur format propre (OWL)
%		\item Afficher l'ontologie sous forme de graphe
%		\item Permettre à l'utilisateur de modifier éléments et liens de
%			l'ontologie
%		\item Sauvegarder l'ontologie dans un fichier, le cloud, ou en imprimer
%			un résumé
%	\end{itemize}
%	}

\subsection{Miscellaneous}
\cventry{Student}{Gardener}{Albanais Home Services}{Moye}{}{Gardening and home services.}

\section{Studies}
\cventry{2012--2015}{Engineering student}{ENSIMAG}{Superior National School of Applied Mathematics and Computer Science}{}{}  % arguments 3 to 6 can be left empty
\cventry{2010--2012}{DUT (two-year technical degree) computer science}{IUT d'Annecy}{}{}{}

\section{Skills}
\cvitem{Low level}{C, assembly (x86, x86\_64, MIPS, kvx), linker, emulation}
\cvitem{Linux}{Kernel and drivers programming}
\cvitem{System}{Linux (Debian derivatives, Redhat derivatives, Arch), FPGA, Bare metal}
\cvitem{Scripting}{Python, Bash, Vim script}
\cvitem{Object}{C++, Java}
\cvitem{Tools}{Git, Make, CMake, LaTex}

\section{Languages}
\cvitem{French}{Native}
\cvitem{English}{Fluent}

\section{Hobbies}
\cvitem{Sports}{Hiking, climbing, skiing}
\cvitem{Programming}{FOSS, Linux enthusiast}
\cvitem{Misc}{Music (Rock/Heavy metal), Video-games}

\end{document}

